% !TeX root = ../main.tex
% Add the above to each chapter to make compiling the PDF easier in some editors.

\chapter{Related Work}\label{chapter:related_work}

There are four topics that are related to this research. 
First, I conducted research on the variety of video filters and how they are implemented in the existing video-conferencing platforms.
we then explored facial extraction technology options that are available. The next section tells us about how we measure synchrony using facial extraction data.
Last section provides an overview of the effectiveness of queue system for real-time data communications.

\section{Video Filters in Video-Conferencing Platforms}

Video filters are created for a wide-range of user needs, such as beautification like smoothing the skin, 
privacy like blurring the background or facial recognition for observing user emotions. By looking at these purposes, 
we then divided the video filters into two categories:
\subsection{Manipulative Filter}
Manipulative filters are used for manipulating facial features, environments or applying augmented reality elements onto the video stream.
These kind of filters are quite well-known from social media users. 
They use them for gaining self-acceptance and engagement with their online audiences~\parencite{ar-filter-on-social-media}.
Professional and privacy concerns are also raised during video calls in a virtual environment so 
blur background filter are built to make the users feel secure of their surroundings~\parencite{blur-privacy-ar}.
There are ton options of manipulative filters that augment the video streams using visual effects or simply just changing the video's color, 
even we are actually possible to implement a customized filter, e.g. applying AR filter on top of another filter.

\subsection{Analysis Filter}
Analysis filters are mostly related to facial and landmark detection of users and used for research and educational purposes. 
These filters are rarely found in commercial video-conferencing platforms, like Zoom, Skype, etc. On the other side, 
open source tools provide analysis filter that can detect faces and landmarks but we can't do it within the video-conferencing platform.
Most of them are standalone toolbox that we need to run separately ~\parencite{tracked}~\parencite{openface}.


\section{Facial Extraction Technology}
Face recognition has gained a lot of attentions in research and business world. Many practical applications could benefit from it, 
such as identification, access control and human-computer interactions which increase the availability of the relevant technologies.
tbc.

\section{Queue Systems}
explain the use of queue systems for real time data applications. list down the available queue systems. why we use rabbitmq instead of other tools.

\section{Synchrony Measurement}
should I add this part?