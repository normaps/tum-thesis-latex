\chapter{\abstractname}

Facial extraction data plays a crucial role in synchrony measurement. 
There are open source tools available now for extracting Facial Action Units (FAU) data,
 such as OpenFace and PyFeat.
However, it is still difficult to run a user study and have a real-time analysis of Facial Action Units (FAU) data
at the same time because the tools are built either as a standalone application 
or a library for offline usage.
Currently, we are building a video-conferencing platform that 
initiates the implementation of analysis filter
for research purposes which still has a lot of rooms for improvement, 
such as low running fps, mixed frames and data lost.
This thesis explores the optimization of facial extraction tool integration 
to the video-conferencing platform by utilizing OpenFace as a case study.
The methodology involves implementing RabbitMQ as a robust queue system to manage and process facial data in real-time.
This thesis also introduces a feature that 
enabled user to have the facial extraction data after running an experiment in the platform called post-processing.
Finally, the real-time data was compared with the post-processing one 
and it showed that they have similar values.
The network and memory measurement was also conducted and confirmed that it improved the real-time analysis filter performance.

